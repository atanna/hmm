\documentclass[10pt,a4paper]{article}
\usepackage[utf8]{inputenc}
\usepackage[english,russian]{babel}
\usepackage{amsmath}
\usepackage{amsfonts}
\usepackage{amssymb}
\usepackage{textcomp}
\begin{document}

\section{Введение}

Дезоксирибонуклеиновая кислота (ДНК) является своеобразным кодом жизни. Эта молекула хранит и передает генетическую программу развития и функционирования живого организма. 
В то же время, все функции ДНК зависят от ее соединений с белками. 
Поэтому, изучение ДНК-белковых взаимодействий актуально и привлекательно.
Chip-seq (chromatin immunoprecipitation - sequencing)\cite{Chip-seq} является одним из современных методов, позволяющим выделить участки ДНК связанные с конкретным белком (одинаково применим к разным белкам). 
Однако, по понятным причинам (сложный биологический эксперимент), погрешность данного метода не может быть нулевой, и безрассудная вера ему лишена смысла. По этому, обычно, к результатам подобных методов накладывается вероятностная модель. Конечно, это добавляет ряд существенных ограничений. Однако, в качестве неоспоримого плюса можно привести тот факт, что хорошо подобранная модель позволяет понять природу данных и изучить их свойства.

Итого, нашей задачей является нахождение модели по последовательности чисел выдаваемых Chip-seq. 

В настоящее время, в качестве семейства искомых моделей, активное приминение находит HMM (Hidden Markov Model)\cite{HMM} второго порядка с Пуассоновским испусканием. 
Данное семейство допускает предположение о том, что каждое состояние (наличие/отсутствие белка в заданной части генома) завист только от одного предыдущего.
Можно ограничиться и более лояльным допущением о том, что состояние зависит от $n$ предыдущих состояний, однако такое допущение резко увеличивает сложность модели ($O(2\textsuperscript{n})$ параметров). Также, сложность заключается в подборе этого $n$ и переобучении в случае, если не все состояния имеют одинаковые длины контекстов зависимости.
Последннее замечание подводит к идее использования VOHMM (Variable Order Hidden Markov Model)\cite{Wang}


\section{Скрытые марковские модели переменного порядка}


 
 
 
\begin{thebibliography}{9}
\bibitem{Wang} Y. Wang, “The variable-length hidden markov model and its applications on sequential data mining” (2005)
\bibitem{Dumont} T. Dumont "Context Tree Estimation in Variable Length
Hidden Markov Models", Université de Paris sud XI
\bibitem{VLMC} P. Buhlmann, A.J.Wyner,"Variable Length Markov Chains"
\bibitem{Chip-seq} \cite{http://bioinformaticsinstitute.ru/projects/563}
\bibitem{HMM} Lawrence R. Rabiner "A tutorial on hidden Markov models and selected applications in speech recognition" (1989)
\end{thebibliography}
\end{document}